\documentclass[a4paper,12pt]{article}
\usepackage[T1]{fontenc}
\usepackage[dvipsnames]{xcolor}
\usepackage[utf8x]{inputenc}
\usepackage[russian]{babel}
\usepackage{pdflscape}
\usepackage{fancyvrb}
\usepackage{geometry}
\usepackage{indentfirst}
\usepackage{wrapfig}
\usepackage{placeins}
\usepackage{graphicx}
\usepackage{seqsplit}
 \geometry{
 a4paper,
 total={210mm,297mm},
 left=20mm,
 right=20mm,
 top=20mm,
 bottom=20mm,
 }

\usepackage{listings}
\lstset{basicstyle=\ttfamily,
  showstringspaces=false,
  commentstyle=\color{OliveGreen},
  keywordstyle=\color{MidnightBlue},
  extendedchars=\true,
  inputencoding=utf8x,
  breaklines=true,
  basicstyle=\ttfamily\footnotesize
}
\usepackage{hyperref}

\RecustomVerbatimCommand{\VerbatimInput}{VerbatimInput}%
{fontsize=\footnotesize,
 %
 %frame=lines,  % top and bottom rule only
 %framesep=2em, % separation between frame and text
 rulecolor=\color{Gray},
 %
 %label=\fbox{\color{Black}text2},
 %labelposition=topline,
}
\begin{document}

\begin{titlepage}
  \begin{center}
    \large
    Міністерство освіти і науки України
    
    Національний технічний університет України

    \textit{“Київський політехнічний інститут”}
    
    Фізико-технічний інститут
    \vspace{5cm}

    \textsc{Розрахункова робота}\\[5mm]
    \textit{з дисципліни}\\
    
    {\LARGE Симетрична криптографія }\\
    {\large Варіант 18}
  \bigskip
    
    
\end{center}
\vspace{3cm}
\hfill
\begin{minipage}{0.3\textwidth}
\large
  Виконав:\\
  \textit{Грубіян Є.О.}\\
  Прийняв:\\
  \textit{Яковлєв С.В.}\\ 
  
\end{minipage}
\bigskip

\vfill
\vfill
\vfill
\begin{center}
  Київ 2016 р.
\end{center}

\end{titlepage}
\section*{Вступ}
Тема розрахункової роботи - дослідження криптографічних властивостей булевих функцій. У запропонованому варіанті пропонуються дві булеві функції для дослідження, а саме: \( f(x) = x^{16257} \) та \( g(x) = x^{16256} \) , де \( x \in GF(2^{15}) \). В якості полінома генератора поля \( GF(2^{15}) \) використовується \( p(x) = x^{15} + x + 1 \).

Програмний код розрахункової роботи написаний на мові C/C++, в силу того, що робота містить багато складних у обчислювальному плані процедур над полем \( GF(2^{15}) \), тому швидкодія є важливим аспектом. Вихідний код також доступний на ресурсі GitHub: \url{https://github.com/juja256/boolean}. Всі вказані параметри для двох булевих функцій в роботі обчислюються близько 20 хвилин на процесорі Intel Core i3-2310m @ 2.10 GHz x 2. Програма тестувалась під ОС Windows 10(Visual Studio 2015) та Linux Mint 17.3(GCC 4.8.4, рівень оптимізації -O3).

\section*{Програмний код}
\lstinputlisting[language=c++, caption=core.h]{core.h}
\bigskip
\lstinputlisting[language=c++, caption=algs.cpp]{algs.cpp}
\bigskip
\lstinputlisting[language=c++, caption=bool\_vec.cpp]{bool_vec.cpp}
\bigskip
\lstinputlisting[language=c++, caption=bool\_func.cpp]{bool_func.cpp}
\bigskip
\lstinputlisting[language=c++, caption=main.cpp]{main.cpp}
\bigskip

\section*{Звіт}
Звіт містить всі криптографічні характеристики, які потрібно було обчислити та приблизний час виконання кожної процедури через вертикальну риску. 
\lstinputlisting[caption=output.txt]{output.txt}

\section*{Висновок}
В ході розрахункової роботи було досліджено, що найкращими криптографічними параметрами володіє функція \( f(x) = x^{16257} \). Оскільки вона має хороші показники нелінійності(вона практично досягає максимального значення на кожній координатній функції) та володіє кореляційним імунітетом 0-го рівня за кожною координатною функцією, тобто є збалансованою за кожною координатною функцією. Разом з тим вона має низький максимум диференціальної ймовірності, всього 1/16384, тобто є порівняно стійкою до диференціального криптоаналізу. Хоча лавинні ефекти відсутні, відхилення коефіціентів розповсюдження помилок від середнього значення не є дуже великими, тобто дана функція володіє "майже" лавинним ефектом. Друга ж функція \( g(x) = x^{16256} \) також є непоганою, але все ж гіршою за першу. 

Найбільш трудомісткою процедурою виявилась процедура знаходження максимальної диференціальної імовірності, близько 560 секунд(10хв.), інші ж параметри  обчислюються відносно швидко: в сумі для двох функцій всі параметри крім MDP займають менше хвилини.

Отже, на мою думку, для криптографічних цілей функція \( f(x) = x^{16257} \) підходить добре.
\end{document}
